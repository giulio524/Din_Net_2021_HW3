\begin{alphaparts}
\questionpart %--------A--------------
Siano \(\mathcal{V} = \{1, \dots n\}\) gli utenti della rete stradale e sia \(\mathcal{R}\) lo spazio delle azioni.La scelta della strada da parte di ciascun utente determina una configurazione \(x \in \mathcal{R}^n\) del gioco in cui le utility sono definite nel modo seguente:
\[
    u_i (x_i, x_{- i}) = - (\tau_{x_i}(z_{x_i}) + z_{x_i}\tau'(z_{x_i})) = : U(x_{i}, z).\]

Il vettore di flusso sulle strade \(z \in \mathcal{P}(\mathcal{R})\) rappresenta in questo caso il \textit{type} del grafo. Si può inoltre notare dalla forma di \(u_i\) come esso sia solamente funzione dell'azione dell' \(i\)-esimo giocatore e del tipo del grafo, pertanto il gioco che stiamo considerando è un gioco \textit{anonimo}. Se si considera una \textit{Noisy Best Response Dynamics} sul grafo, i rate di transizione sono definiti come:
\begin{equation*}
    \Theta_{ik}(z) = \frac{e^{\beta U(k, z)}}{ \sum \limits_{j \in \mathcal{R}}^{} e^{\beta U(j, z)}},
\end{equation*} 

mentre il \textit{Drift Operator} \(F(z)\) può essere calcolato come segue:

\[
F_i(z) = (z_1 \Theta_{1i}(z) + \dots z_k \Theta_{ki}(z) ) - (z_i \Theta_{i1}(z) + \dots + z_i \Theta_{ik}(z) ) =
\]
\[ 
 = \left(\sum \limits_{j \in \mathcal{R}}^{} z_j \Theta_{ji}(z) \right) - z_i \left( \sum \limits_{j \in \mathcal{R}}^{} \Theta_{ij}(z)\right) =    
\]
\[
    = \left(\frac{e^{\beta U(i, z)}}{ \sum \limits_{j \in \mathcal{R}}^{} e^{\beta U(j, z)}}  \sum \limits_{j \in \mathcal{R}}^{} z_j \right) - z_i \left( \frac{\sum \limits_{j \in \mathcal{R}}^{} e^{\beta U(j, z)}}{ \sum \limits_{j \in \mathcal{R}}^{} e^{\beta U(j, z)}}\right) = g_i(z) - z_i.
\]
Pertanto \(g(z)\) è una funzione così definita:

\begin{equation*}
    g_i(z) = \frac{\exp [\beta - (\tau_{i}(z_{i}) + z_{i}\tau'(z_{i})) ]}{ \sum \limits_{j \in \mathcal{R}}^{} \exp [\beta - (\tau_{j}(z_{j}) + z_{j}\tau'(z_{j})) ]} \quad \forall i \in \mathcal{R}
\end{equation*}

\questionpart %------B----------------
Siano
\[
    \Phi(z) =  \sum \limits_{i}^{} z
    _i \tau_i(z_i) \quad \quad H(z) = -  \sum \limits_{i}^{} z_i \log z_i\]

    Utilizzando le nozioni afferenti alla sezione 11.3.2 delle \textit{lecture notes} si può notare che l'esistenza della funzione \( \hat{\Phi}(z) := - \Phi(z)\) dimostra che il gioco è \textit{potenziale}. Notiamo inoltre che, essendo le funzioni di delay \(\tau_i(z_i)\) convesse allora \(\hat{\Phi}(z)\) è concava. Possiamo quindi utilizzare il lemma 11.1 delle \textit{lecture notes} che ci assicura che tutti i punti del simplesso \(\mathcal{P}(\mathcal{R})\) che rappresentano un equilibrio del limite idrodinamico sono punti stazionari della funzione 
    \[
     \hat{V}_\beta(z) := - V_\beta(z) = \hat{\Phi}(z) + \frac{1}{\beta}H(z).   
    \]
    Essendo \(\hat{V}_\beta \) strettamente concava, essa ha un unico punto stazionario. Trovarlo equivale a risolvere il problema:
    \begin{equation*}
        z^\beta = \arg\min \limits_{z \in \mathcal{P}(\mathcal{R})} V_\beta(z)
    \end{equation*}
    che può essere risolto ponendo \(\nabla V_\beta(z) = 0 \).
    In particolare:
    \[
    \frac{\delta V_\beta(z)}{\delta z_k} = \tau_{k}(z_{k}) + z_{k}\tau'(z_{k})  + \frac{1}{\beta}(\log(z_k)+ 1).  
    \]
    Pertanto \(z^\beta\) soddisfa:
    \[
        V_\beta(z^\beta) = 0 \iff \tau_{k}(z_{k}^\beta) + z_{k}^\beta\tau'(z_{k}^\beta)\, = - \frac{1}{\beta}(\log(z_k^\beta)+ 1) \quad \forall k \in \mathcal{R}. 
    \]
    In particolare
    \[
        g_k(z^\beta) = \frac{z_k^\beta e^1}{e^1  \sum \limits_{h}^{} z^\beta_h} = z_k \quad \quad \forall k \in \mathcal{R}\]
    Pertanto \(z^\beta\) è un punto di equilibro stabile per la ODE della dinamica \textit{mean field}.
\end{alphaparts}