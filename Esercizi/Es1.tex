\begin{alphaparts}
    %-------------------------------------A---------------------------------------
   \questionpart
   Date le seguenti funzioni di utilità per ciascun giocatore:
   \[
    u_i(x)=- \frac{x_i^2}{2}+ c_i x_i + \beta \sum\limits_{j \neq i} W_{ij}x_j x_i 
    \]
    trovare le funzioni di \textit{Best Response} per ciascun giocatore significa trovare le seguenti:

    \[BR_i(x_{ - i}) = \arg\max_{x_i \in \mathcal{A}} u_i(x_i, x_{ - i}).\]

    Poiché \(u_i(x)\) è concava rispetto a \(x_i\), trovare punti stazionari rispetto a \(x_i\) significa trovare massimi globali:
    \[
        \frac{\delta u_i ( x )}{\delta x_i} = -x_i + c_i + \beta  \sum \limits_{j \neq i}^{} W_{ij} x_j
    \]
    \[
        \frac{\delta u_i ( x )}{\delta x_i} = 0 \quad \iff \quad x_i = c_i + \beta  \sum \limits_{j \neq i}^{} W_{ij} x_j
    .\]
    Pertanto:
    \[
        BR_i(x_{-i}) = c_i + \beta  \sum \limits_{j \neq i}^{} W_{ij}x_j 
    .\]

    %------------------------------B----------------------------------------------
    \questionpart
    Poichè il grafo è privo di self-loop, la matrice di adiacenza \(W\) non presenta elementi non nulli sulla diagonale. è pertanto possibile utilizzare la seguente espressione per gli equilibri di Nash:
    \begin{equation}\label{eq_nash}
        x^* = \beta W x^* + c
    .\end{equation}
    Quindi, in forma esplicita, abbiamo:
    \begin{equation*}
        x^* = (I -\beta W)^{-1} c
    .\end{equation*}
    Questo significa che esistenza ed unicità di eventuali equilibri di Nash dipendono dalla singolarità della matrice \((I - \beta W)\): se l'inversa della matrice \((I - \beta W)\) esiste ed è unica, allora esiste ed è unico l'equilibrio di Nash \(x^*\) associato .

    DA INDAGARE SE ESISTONO CASI IN CUI L'EQUILIBRIO NON è UNICO

    La condizione:
    \begin{equation}\label{eq_condition_beta}
        \beta \omega_i < 1 \quad \quad \quad \forall i \in \mathcal{V}
    \end{equation}
    impone che le righe della matrice \(\beta W\) sommino ad una quantità strettamente minore di 1. La matrice \(\beta W\) si dice pertanto essere \textit{substocastica} e il suo raggio spettrale soddisfa la condizione \(\rho(\beta W) < 1\). Può quindi essere applicato il \textit{Criterio di Neumann} che ci assicura esistenza ed unicità della matrice \((I - \beta W)^{-1}\). In particolare vale:
    \begin{equation}
        (I-\beta W)^{-1} =  \sum \limits_{k=0}^{\infty} \beta^k W^k
    \end{equation}

    %-----------------------------C----------------------------------
    \questionpart
    Come visto per il punto precedente, se esiste ed è unico, l'equilibrio di Nash del gioco è dato da:
    \[
        x^* = (I -\beta W)^{-1} c
        \]
    pertanto è nella forma \(x^* = Mc\) e dipende linearmente dal vettore c. Nel caso in cui valga \ref{eq_condition_beta} allora, per il \textit{criterio di Neumann} abbiamo:
    \begin{equation}
        M =  \sum \limits_{k=0}^{\infty} \beta^k W^k
    .\end{equation}
    Inoltre vale:
    \begin{equation*}
        (W^k)_{ij} = \text{ \# di path da i a j di lunghezza k} \implies (W^k)_{ij} \geq 0 \quad \forall i, j \in \mathcal{V}
    \end{equation*}
    Pertanto \(M\) è dato dalla somma di matrici con elementi tutti non negativi:
    \[M_{ij} \geq 0 \quad \quad \forall i, j \in \mathcal{V}\]

    %-----------------------------D-------------------------------------------
    \questionpart
    
\end{alphaparts}