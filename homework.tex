\documentclass[11pt,largemargins]{homework}

\newcommand{\hwname}{Giulio Nenna}
\newcommand{\hwemail}{s292399@studenti.polito.it}
\newcommand{\hwtype}{Homework}
\newcommand{\hwnum}{3}
\newcommand{\hwclass}{}
\newcommand{\hwlecture}{}
\newcommand{\hwsection}{}

% This is just used to generate filler content. You don't need it in an actual
% homework!
\usepackage{lipsum}
\usepackage{amssymb}
\usepackage[utf8]{inputenc}
\usepackage[T1]{fontenc}
\usepackage{lmodern}
\usepackage{amsfonts}
\usepackage{hyperref}
\usepackage{bbm}
\usepackage{amsmath}
\usepackage{mcode}
\usepackage{relsize}
\usepackage{epstopdf}
\usepackage{subcaption}
\usepackage{float}
\usepackage{bbm}
\usepackage{amsthm}
\usepackage{mathtools}
\usepackage{tikzit}
\usepackage{caption}
\usepackage{listings}
\usepackage{color} %red, green, blue, yellow, cyan, magenta, black, white
\definecolor{mygreen}{RGB}{28,172,0} % color values Red, Green, Blue
\definecolor{mylilas}{RGB}{170,55,241}
\input{Stilegrafo.tikzstyles}
\begin{document}


\lstset{language=Matlab,%
    %basicstyle=\color{red},
    breaklines=true,%
    morekeywords={matlab2tikz},
    keywordstyle=\color{blue},%
    morekeywords=[2]{1}, keywordstyle=[2]{\color{black}},
    identifierstyle=\color{black},%
    stringstyle=\color{mylilas},
    commentstyle=\color{mygreen},%
    showstringspaces=false,%without this there will be a symbol in the places where there is a space
    numbers=left,%
    numberstyle={\tiny \color{black}},% size of the numbers
    numbersep=9pt, % this defines how far the numbers are from the text
    emph=[1]{for,end,break},emphstyle=[1]\color{red}, %some words to emphasise
    %emph=[2]{word1,word2}, emphstyle=[2]{style},    
}
\maketitle

\begin{center}
  Realizzato in collaborazione con Alessandro Bonaduce (s289906@studenti.polito.it), Davide Grande (s292174@studenti.polito.it).
\end{center}


\section{}% --------------ESERCIZIO 1------------------------------
\begin{alphaparts}
    %-------------------------------------A---------------------------------------
   \questionpart
   Date le seguenti funzioni di utilità per ciascun giocatore:
   \[
    u_i(x)=- \frac{x_i^2}{2}+ c_i x_i + \beta \sum\limits_{j \neq i} W_{ij}x_j x_i 
    \]
    trovare le funzioni di \textit{Best Response} per ciascun giocatore significa trovare le seguenti:

    \[BR_i(x_{ - i}) = \arg\max_{x_i \in \mathcal{A}} u_i(x_i, x_{ - i}).\]

    Poiché \(u_i(x)\) è concava rispetto a \(x_i\), trovare punti stazionari rispetto a \(x_i\) significa trovare massimi globali:
    \[
        \frac{\delta u_i ( x )}{\delta x_i} = -x_i + c_i + \beta  \sum \limits_{j \neq i}^{} W_{ij} x_j
    \]
    \[
        \frac{\delta u_i ( x )}{\delta x_i} = 0 \quad \iff \quad x_i = c_i + \beta  \sum \limits_{j \neq i}^{} W_{ij} x_j
    .\]
    Pertanto:
    \[
        BR_i(x_{-i}) = c_i + \beta  \sum \limits_{j \neq i}^{} W_{ij}x_j 
    .\]

    %------------------------------B----------------------------------------------
    \questionpart
    Poichè il grafo è privo di self-loop, la matrice di adiacenza \(W\) non presenta elementi non nulli sulla diagonale. è pertanto possibile utilizzare la seguente espressione per gli equilibri di Nash:
    \begin{equation}\label{eq_nash}
        x^* = \beta W x^* + c
    .\end{equation}
    Quindi, in forma esplicita, abbiamo:
    \begin{equation*}
        x^* = (I -\beta W)^{-1} c
    .\end{equation*}
    Questo significa che esistenza ed unicità di eventuali equilibri di Nash dipendono dalla singolarità della matrice \((I - \beta W)\): se l'inversa della matrice \((I - \beta W)\) esiste ed è unica, allora esiste ed è unico l'equilibrio di Nash \(x^*\) associato .

    DA INDAGARE SE ESISTONO CASI IN CUI L'EQUILIBRIO NON è UNICO

    La condizione:
    \begin{equation}\label{eq_condition_beta}
        \beta \omega_i < 1 \quad \quad \quad \forall i \in \mathcal{V}
    \end{equation}
    impone che le righe della matrice \(\beta W\) sommino ad una quantità strettamente minore di 1. La matrice \(\beta W\) si dice pertanto essere \textit{substocastica} e il suo raggio spettrale soddisfa la condizione \(\rho(\beta W) < 1\). Può quindi essere applicato il \textit{Criterio di Neumann} che ci assicura esistenza ed unicità della matrice \((I - \beta W)^{-1}\). In particolare vale:
    \begin{equation}
        (I-\beta W)^{-1} =  \sum \limits_{k=0}^{\infty} \beta^k W^k
    \end{equation}

    %-----------------------------C----------------------------------
    \questionpart
    Come visto per il punto precedente, se esiste ed è unico, l'equilibrio di Nash del gioco è dato da:
    \[
        x^* = (I -\beta W)^{-1} c
        \]
    pertanto è nella forma \(x^* = Mc\) e dipende linearmente dal vettore c. Nel caso in cui valga \ref{eq_condition_beta} allora, per il \textit{criterio di Neumann} abbiamo:
    \begin{equation} \label{eq_neumann}
        M =  \sum \limits_{k=0}^{\infty} \beta^k W^k
    .\end{equation}
    Inoltre vale:
    \begin{equation*}
        (W^k)_{ij} = \text{ \# di path da i a j di lunghezza k} \implies (W^k)_{ij} \geq 0 \quad \forall i, j \in \mathcal{V}
    \end{equation*}
    Pertanto \(M\) è dato dalla somma di matrici con elementi tutti non negativi:
    \[M_{ij} \geq 0 \quad \quad \forall i, j \in \mathcal{V}\]

    %-----------------------------D-------------------------------------------
    \questionpart
    Il vettore di centralità di Katz del grafo è dato da:
    \begin{equation*}
        z = \left(I - \left(\frac{1-\beta_K}{\lambda_W}\right)W'\right)^{-1}\beta_K \mu
    \end{equation*}
    Abbiamo quindi il seguente risultato:
    \[
    z'c = \left(\left(I - \frac{1 - \beta_K}{\lambda_W}W'\right)^{-1} \beta_K \mu\right)' c = \beta_k \mu' \left(I - \frac{1 - \beta_K}{\lambda_W}W\right)^{-1}c    
    \]
    Ponendo quindi \footnote{L'operazione è lecita dal momento che  la matrice \(\beta W\) è substocastica. Questo vuol dire che \(\rho(\beta W) = \beta \rho(W) < 1\), pertanto \(\beta_K \in [0,1]\).}
    \[ \frac{1 - \beta_K}{\lambda_W} = \beta \iff \beta_K = 1 - \lambda_W \beta \]
    e
    \[\mu = \frac{1}{1 - \lambda_W \beta} \mathbbm{1}\]
    si ottiene un vettore di centralità di Katz:
    
    \begin{equation}\label{eq_zhat}
        \hat{z} =\left(I - \beta W'\right)^{-1}\mathbbm{1}
    \end{equation}
    tale per cui:
    
    \begin{equation}\label{eq_yscalar}
        \hat{z}'c = \mathbbm{1}' (I - \beta W)^{-1} c =  \sum \limits_{j \in \mathbb{V}}^{} x^*_j = y.
    \end{equation}

    %-------------------------E---------------------------------------------
    \questionpart
    Vale il seguente risultato:
    \[
    Var(y) = Var(\hat{z}'c) = \hat{z}'VarCov(c)\hat{z}    
    \]
    Essendo il vettore \(c\) un vettore di variabili aleatorie indipendenti:
    \[
    Varcov(c) = \begin{bmatrix}
        \sigma_1^2 & & 0 \\
         & \ddots & \\
         0 & & \sigma_n^2
    \end{bmatrix}    
    \]
    Pertanto vale:
    \[Var(y) =  \sum \limits_{i = 1}^{n} \hat{z}_i^2 \sigma_i^2 =  \sum \limits_{i = 1}^{n} ((I - \beta W')^{-1}\mathbbm{1})_i^2 \sigma_i^2\]

    %-----------------------F-------------------------------------------------
    \questionpart
    Sia \(i_K\) il key player del gioco e sia \(W^{(- i_K)}\) la matrice di adiacenza del grafo \(\mathcal{G}^{(- i_K)}\) privato del nodo \(i_K\). Poiché la funzione di utility di ciascun nodo rimane invariata nella forma, si possono applicare gli stessi ragionamenti dei punti precedenti per trovare che esiste ed è unico l'equilibrio di Nash:
    \[
    x^{*(- i_K)} = (I- \beta W^{(- i_K)})^{- 1}c  
    \]
    per tutte le \(\beta\) che soddisfano \ref{eq_condition_beta} dal momento che la matrice \(\beta W^{(- i_K)}\) risulta substocastica. Se definiamo \(M^{(- i_K)} = (I- \beta W^{(- i_K)})^{- 1}\) allora, poiché \(c= \mathbbm{1}\), abbiamo: 
    \begin{equation}    
            x^{*(- i_K)} = M^{(- i_K)} \mathbbm{1}
    \end{equation}
    
    %-----------------G-----------------------------------------------------
    \questionpart
    Per ogni \(k \neq i \neq j\) vale:
    \begin{gather*}
        M_{ij}M_{ik} = M_{ii}(M_{jk}-M_{jk}^{(-i)}).
    \end{gather*}

    %--------------------------H--------------------------------------------
    \questionpart
    Poichè il gioco è simmetrico, da \ref{eq_zhat}, si ottiene che 
    \begin{equation}
        \hat{z} =\left(I - \beta W\right)^{-1}\mathbbm{1} = M\mathbbm{1}
    \end{equation}
    Notiamo inoltre da \ref{eq_neumann} che se il gioco è simmetrico allora anche la matrice \(M\) sarà simmetrica. Utilizzando quindi \ref{eq_yscalar} si ottiene:
    \begin{equation}
        y = \hat{z}' c = (M\mathbbm{1})'\mathbbm{1} = \mathbbm{1}' M \mathbbm{1} =  \sum \limits_{j}  \sum \limits_{k}^{} M_{jk} 
    \end{equation}
    e, analogamente
    \[
    y^{(- i)} = \mathbbm{1}' M^{(- i)} \mathbbm{1} = \sum \limits_{j\neq i}  \sum \limits_{k \neq i}^{} M_{jk}^{(- i)}
    .\]
    A questo punto si ottiene:
    \[
    y- y^{(- i)} =   \sum \limits_{j}  \sum \limits_{k}^{} M_{jk} - \sum \limits_{j\neq i}  \sum \limits_{k \neq i}^{} M_{jk}^{(- i)} =
    \]
    \[
    =  \left(\sum \limits_{j\neq i}  \sum \limits_{k \neq i}  M_{jk} -  M_{jk}^{(- i)} \right)+  \left(\sum \limits_{k \neq i}^{} M_{ik}\right) +  \left(\sum \limits_{j\neq i}^{} M_{ji} \right)+ M_{ii}
    \]
    Utilizzando il punto (g): \( (M_{jk}-M_{jk}^{(-i)}) = (M_{ij}M_{ik})/M_{ii}\) pertanto \footnote{Questo passaggio è sempre possibile in quanto \(M_{ii} > 0 \quad \forall i \in \mathcal{V}\). Infatti, essendo il grafo simmetrico, abbiamo che \((W^2)_{ii} > 0 \quad \forall i \in \mathcal{V}\). Il risultato segue dalla definizione di \(M\) e dal risultato ottenuto attraverso il criterio di Neumann. }

    \[
    y - y^{(- i)} = \left(\sum \limits_{j\neq i}  \sum \limits_{k \neq i} \frac{M_{ij}M_{ik}}{M_{ii}}\right) +   2 \left(  \sum \limits_{j \neq i}^{} M_{ij}\right) + M_{ii} =   
    \]
    \[
    = \frac{1}{M_{ii}} \left(\sum \limits_{j\neq i}  \sum \limits_{k \neq i} M_{ij}M_{ik}\right) + 2 \left(  \sum \limits_{j}^{} M_{ij}\right) - M_{ii} =
    \]
    \[
        =\frac{1}{M_{ii}} \left(  \sum \limits_{j\neq i}^{} M_{ij}  \sum \limits_{k}^{} M_{ik}- M_{ii} \sum \limits_{j\neq i}^{} M_{ij}\right) + 2 \left(  \sum \limits_{j}^{} M_{ij}\right) - M_{ii}
    \]
    Ora, poiché \(z_i =  \sum \limits_{j}^{} M_{ij}\) e \(  \sum \limits_{j \neq i}^{} M_{ij} = z_i - M_{ii}\), otteniamo:
    \[
        y - y^{(- i)} =  \frac{1}{M_{ii}} \left( \sum \limits_{j\neq i}^{} M_{ij} (z_i - M_{ii})\right) + 2z_i - M_{ii} = 
    \]
    \[
    =   \frac{1}{M_{ii}} (z_i - M_{ii})^2 + 2z_i - M_{ii} = \frac{z_i^2}{M_{ii}}  
    \]
    Pertanto il \textit{key player} è quel giocatore \(i\) che massimizza il rapporto \(\frac{z_i^2}{M_{ii}}.\)

\questionpart %-------------- i ---------------------------------
Per il grafo in esame esistono tre gruppi di simmetria:
\begin{gather*}
    \mathcal{V}_1 = \{ 1,2,3,9,10,11\} \\
    \mathcal{V}_2 = \{ 4,5,7,8\}\\
    \mathcal{V}_3 = \{6\}.
\end{gather*}
L'obiettivo è quello di risolvere il seguente sistema lineare:
\begin{equation} \label{eq_sus}
    z = \beta W' z + \mathbbm{1}
\end{equation}
Utilizzando i gruppi di simmetria è possibile ricavare che:
\begin{gather*}
    z_i = \lambda \in \mathbb{R}^+ \quad \quad \forall i \in \mathcal{V}_1\\
    z_i = s \in \mathbb{R}^+ \quad \quad \forall i \in \mathcal{V}_2\\
    z_6 = \theta \in \mathbb{R}^+ 
\end{gather*}

Pertanto basterà ricavare tre equazioni linearmente indipendenti dal sistema \ref{eq_sus} per trovare \(z\). Ad esempio dalla prima riga si ottiene:
\[z_1 = \beta(z_2+ z_3+ z_4+ z_5)+ 1 \implies (1- 2\beta)\lambda = 2\beta s + 1\]
mentre dall'equazione sul nodo 6 abbiamo:
\[z_6 = \beta(z_4 + z_5 + z_7 + z_8) + 1 \implies \theta = 4\beta s + 1.\]
Infine, dall'equazione sul nodo 4 si ottiene:
\[z_4 = \beta (z_1 + z_2 + z_3 + z_5 + z_6) + 1 \implies (1- \beta) s = \beta (3 \lambda + \theta) + 1.\]
Utilizzando queste tre equazioni si ottiene il sistema ridotto:
\[
\begin{bmatrix}
     1- 2\beta & - 2\beta & 0 \\
     0 & - 4\beta & 1 \\
    - 3\beta\lambda & 1- \beta & - \beta
\end{bmatrix}
\begin{bmatrix} \lambda \\ s \\ \theta \end{bmatrix} =
\begin{bmatrix} 1 \\ 1 \\ 1 \end{bmatrix}.     
\]

Denominando con il pedice \(a\) i risultati calcolati utilizzando il valore \(\beta = 0.1\) e con il pedice \(b\) quelli per cui \(\beta = 0.2\) si ottiene:
\begin{align*}
    \begin{cases}
      \lambda_a = 1.72 \\
      s_a = 1.87 \\
      \theta_a = 1.75
    \end{cases} && \begin{cases}
      \lambda_b = 7.77 \\
      s_b = 9.16 \\
      \theta_b = 8.33
    \end{cases}.
  \end{align*}

  Per quanto riguarda la matrice \(M = (I- \beta W')^{-1}\), dalle informazioni sulla simmetria è possibile ricavare:
  \begin{gather*}
      M_{ii} = M_\lambda \in \mathbb{R}^+\quad\quad \forall i \in \mathcal{V}_1\\
      M_{ii} = M_s \in \mathbb{R}^+ \quad\quad \forall i \in \mathcal{V}_2\\
      M_{6,6} = M_\theta
      \end{gather*}
    Calcolando si ottiene:
    \begin{align*}
        \begin{cases}
          M_{\lambda_a} = 1.061 \\
          M_{s_a} = 1.076 \\
          M_{\theta_a }= 1.051
        \end{cases} && \begin{cases}
          M_{\lambda_b} = 1.851 \\
          M_{s_b} = 2.083 \\
          M_{\theta_b} = 1.666
        \end{cases}.
      \end{align*}

      A questo punto è possibile individuare i \textit{key player} \(K_a\) e \(K_b\) andando a calcolare \(\arg\max \limits_{i \in \mathcal{V}} \frac{z_i^2}{M_{ii}}\).  Si ottiene:
      
      \begin{gather*}
          K_a = i ,\quad   \frac{z_i^2}{M_{ii}} = 3.2811\quad \forall i \in \mathcal{V}_2 \\
          K_b = 6,\quad \frac{z_6^2}{M_{66}} = 41.666  
      \end{gather*}

\end{alphaparts}


\section{}% --------------ESERCIZIO 2------------------------------
\begin{alphaparts}
\questionpart %--------A--------------
Siano \(\mathcal{V} = \{1, \dots n\}\) gli utenti della rete stradale e sia \(\mathcal{R}\) lo spazio delle azioni.La scelta della strada da parte di ciascun utente determina una configurazione \(x \in \mathcal{R}^n\) del gioco in cui le utility sono definite nel modo seguente:
\[
    u_i (x_i, x_{- i}) = - (\tau_{x_i}(z_{x_i}) + z_{x_i}\tau'(z_{x_i})) = : U(x_{i}, z).\]

Il vettore di flusso sulle strade \(z \in \mathcal{P}(\mathcal{R})\) rappresenta in questo caso il \textit{type} del grafo. Si può inoltre notare dalla forma di \(u_i\) come esso sia solamente funzione dell'azione dell' \(i\)-esimo giocatore e del tipo del grafo, pertanto il gioco che stiamo considerando è un gioco \textit{anonimo}. Se si considera una \textit{Noisy Best Response Dynamics} sul grafo, i rate di transizione sono definiti come:
\begin{equation*}
    \Theta_{ik}(z) = \frac{e^{\beta U(k, z)}}{ \sum \limits_{j \in \mathcal{R}}^{} e^{\beta U(j, z)}},
\end{equation*} 

mentre il \textit{Drift Operator} \(F(z)\) può essere calcolato come segue:

\[
F_i(z) = (z_1 \Theta_{1i}(z) + \dots z_k \Theta_{ki}(z) ) - (z_i \Theta_{i1}(z) + \dots + z_i \Theta_{ik}(z) ) =
\]
\[ 
 = \left(\sum \limits_{j \in \mathcal{R}}^{} z_j \Theta_{ji}(z) \right) - z_i \left( \sum \limits_{j \in \mathcal{R}}^{} \Theta_{ij}(z)\right) =    
\]
\[
    = \left(\frac{e^{\beta U(i, z)}}{ \sum \limits_{j \in \mathcal{R}}^{} e^{\beta U(j, z)}}  \sum \limits_{j \in \mathcal{R}}^{} z_j \right) - z_i \left( \frac{\sum \limits_{j \in \mathcal{R}}^{} e^{\beta U(j, z)}}{ \sum \limits_{j \in \mathcal{R}}^{} e^{\beta U(j, z)}}\right) = g_i(z) - z_i.
\]
Pertanto \(g(z)\) è una funzione così definita:

\begin{equation*}
    g_i(z) = \frac{\exp [\beta - (\tau_{i}(z_{i}) + z_{i}\tau'(z_{i})) ]}{ \sum \limits_{j \in \mathcal{R}}^{} \exp [\beta - (\tau_{j}(z_{j}) + z_{j}\tau'(z_{j})) ]} \quad \forall i \in \mathcal{R}
\end{equation*}

\questionpart %------B----------------
Siano
\[
    \Phi(z) =  \sum \limits_{i}^{} z
    _i \tau_i(z_i) \quad \quad H(z) = -  \sum \limits_{i}^{} z_i \log z_i\]

    Utilizzando le nozioni afferenti alla sezione 11.3.2 delle \textit{lecture notes} si può notare che l'esistenza della funzione \( \hat{\Phi}(z) := - \Phi(z)\) dimostra che il gioco è \textit{potenziale}. Notiamo inoltre che, essendo le funzioni di delay \(\tau_i(z_i)\) convesse allora \(\hat{\Phi}(z)\) è concava. Possiamo quindi utilizzare il lemma 11.1 delle \textit{lecture notes} che ci assicura che tutti i punti del simplesso \(\mathcal{P}(\mathcal{R})\) che rappresentano un equilibrio del limite idrodinamico sono punti stazionari della funzione 
    \[
     \hat{V}_\beta(z) := - V_\beta(z) = \hat{\Phi}(z) + \frac{1}{\beta}H(z).   
    \]
    Essendo \(\hat{V}_\beta \) strettamente concava, essa ha un unico punto stazionario. Trovarlo equivale a risolvere il problema:
    \begin{equation*}
        z^\beta = \arg\min \limits_{z \in \mathcal{P}(\mathcal{R})} V_\beta(z)
    \end{equation*}
    che può essere risolto ponendo \(\nabla V_\beta(z) = 0 \).
    In particolare:
    \[
    \frac{\delta V_\beta(z)}{\delta z_k} = \tau_{k}(z_{k}) + z_{k}\tau'(z_{k})  + \frac{1}{\beta}(\log(z_k)+ 1).  
    \]
    Pertanto \(z^\beta\) soddisfa:
    \[
        V_\beta(z^\beta) = 0 \iff \tau_{k}(z_{k}^\beta) + z_{k}^\beta\tau'(z_{k}^\beta)\, = - \frac{1}{\beta}(\log(z_k^\beta)+ 1) \quad \forall k \in \mathcal{R}. 
    \]
    In particolare
    \[
        g_k(z^\beta) = \frac{z_k^\beta e^1}{e^1  \sum \limits_{h}^{} z^\beta_h} = z_k \quad \quad \forall k \in \mathcal{R}\]
    Pertanto \(z^\beta\) è un punto di equilibro stabile per la ODE della dinamica \textit{mean field}.

    \questionpart
    Il vettore \(z^\beta\) soddisfa la seguente relazione:
    \[
        \tau_{k}(z_{k}^\beta) + z_{k}^\beta\tau'(z_{k}^\beta)\, = - \frac{1}{\beta}(\log(z_k^\beta)+ 1) \quad \forall k \in \mathcal{R}
        \]
    Poiché 
    \[
    \lim_{\beta \to \infty} \frac{1}{\beta} (\log (z^\beta_k) + 1) = 0    
    \]
    Allora, per \(\beta \to \infty\), il vettore \(z^\circ\) soddisfa:

    \[
        \tau_{k}(z_{k}^\circ) + z_{k}^\circ\tau'(z_{k}^\beta)= 0 .   
    \]

    A questo risultato si arriva seguendo lo stesso ragionamento del punto precedente considerando che:
    \[
    z^\circ = \arg\min \limits_{z \in \mathcal{P}\mathcal{R}} \left( \lim_{\beta \to \infty} V_\beta(z)\right) =   \arg\min \limits_{z \in \mathcal{P}\mathcal{R}} \Phi(z)
    \]
    Dal momento che \(\Phi(z) =  \sum \limits_{i}^{} z
    _i \tau_i(z_i)\) allora \(\Phi(z)\) rappresenta l'\textit{average delay} in quanto è dato dalla somma dei delay su ciascuna strada pesati per la frazione di utenti che la utilizzano. Pertanto \(z^\circ\) minimizza l'\textit{average delay} ed è per definizione un \textit{social optimum traffic assignment}.
   
\end{alphaparts}
\newpage
\section{}% --------------ESERCIZIO 3------------------------------

Il fenomeno di trasmissione/guarigione della malattia nel modello SIR può essere modellato come una corsa esponenziale tra variabili aleatorie con distribuzione esponenziale di parametro $\beta$ (che rappresenta l'attivazione dell'arco tra due nodi e quindi l'infezione) e $1$ (che modellizza la mutazione spontanea da infetto a \textit{recovered}).

Prendiamo ora in considerazione il caso del grafo ad anello in cui all'istante iniziale un solo nodo è infetto e tutti gli altri sono suscettibili.
È ragionevole immaginare che non appena la malattia viene trasmessa a un nuovo nodo, dai due nuovi nodi infetti la pandemia propaghi come due dinamiche SIR indipendenti su due grafi linea.
In particolare, nel caso in cui il primo nodo infetti il suo vicino a sinistra (il nodo $n$), i possibili stati assorbenti sono univocamente determinati dal numero di nodi \textit{recovered} in ciascuna delle due dinamiche indipendenti. 
Pertanto, se indichiamo con $l$ il numero di nodi \textit{recovered} della dinamica `a sinistra' del nodo $n$ e con $r$ il numero di \textit{recovered}  `a destra' del nodo iniziale, gli stati assorbenti sono nella forma:
\[
x_{r,l} = \{ R,  \underbrace{(R,...,R)}_{r \text{ volte}},  \underbrace{(S,...,S)}_{n-r-l-2 \text{ volte}},  \underbrace{(R,...,R)}_{l \text{ volte}}, R \} \quad \quad \forall l,r \in \mathbb{N}: l+r \leq n-2.
 \]
Nel caso in cui il primo nodo infetti il suo vicino a destra gli stati assorbenti sono nella forma: 
\[
y_{r,l} = \{ R,R,  \underbrace{(R,...,R)}_{r \text{ volte}},  \underbrace{(S,...,S)}_{n-r-l-2 \text{ volte}},  \underbrace{(R,...,R)}_{l \text{ volte}} \} \quad \quad \forall l,r \in \mathbb{N}: l+r \leq n-2.
 \]
Osserviamo che, in base a questa definizione, alcuni di questi stati assorbenti si sovrappongono. Infatti:
\[
x_{r,l} = y_{r- 1, l + 1} \quad \forall r,l \geq 1.     
\]
 Inoltre due altri possibili stati assorbenti sono:
 \[
 x_0 = \{ R, S, \dots , S\},
 \]
 \[
 x_n = \{R,R, \dots , R\} .       
 \]

Poiché in un grafo ad anello ogni nodo ha esattamente due vicini la probabilità che il primo nodo infetti uno dei suoi vicini è pari alla probabilità che la corsa esponenziale tra la guarigione e i due edge sia vinta da uno dei due edge.
Questa probabilità è dunque pari a
    \[
	\frac{\beta}{1+2\beta}.
        \]
A questo punto, utilizzando i risultati noti per il modello SIR su linea, si può calcolare la probabilità che la pandemia prosegua e si arresti dopo aver colpito altri $r$ nodi sulla linea `a destra' e $l$ nodi sulla linea `a sinistra' dei due infetti iniziali.
    \[
	\frac{\beta^l}{(1+\beta)^{l+1}}\frac{\beta^r}{(1+\beta)^{r+1}}
        \]
Poiché i numeri $r$ ed $l$ identificano univocamente uno stato assorbente, la probabilità di ingresso si calcola semplicemente come
 \[
	\mathbb{P}(X(\infty)=y_{r,l} | X(0))  =\mathbb{P}(X(\infty)=x_{r,l} | X(0))  =\frac{\beta}{1+2\beta}\frac{\beta^l}{(1+\beta)^{l+1}}\frac{\beta^r}{(1+\beta)^{r+1}}
        \]
       
       In generale possiamo calcolare la probabilità di finire in uno stato assorbente qualsiasi tale per cui sono presenti $K$ \textit{recovered} rimanenti al termine della pandemia. Poiché non distinguiamo quale dei due nodi (a destra o a sinistra) sia stato infettato per primo la probabilità di ottenere $k$ \textit{recovered} alla fine della pandemia si scrive come
        \[
	\mathbb{P}(K=k)=\sum_{l+r=k-2} \mathbb{P}(X(\infty)=y_{r,l} | X(0))  +\mathbb{P}(X(\infty)=x_{r,l} | X(0))=
        \]

                \[
	=\mathlarger{\mathlarger{\sum}}_{l+r=k-2}\frac{2\beta}{1+2\beta}\frac{\beta^l}{(1+\beta)^{l+1}}\frac{\beta^r}{(1+\beta)^{r+1}} =
        \]
          \[
	=(k-1)\frac{2\beta}{1+2\beta}\frac{\beta^{k-2}}{(1+\beta)^k}=
        \]
         \begin{equation}\label{eq_1}
	=2(k-1)\frac{1}{1+2\beta}\frac{\beta^{k-1}}{(1+\beta)^k}
	\end{equation}
       

dove in \eqref{eq_1} il termine $(k-1)$ indica il numero di modi in cui possiamo ottenere la quantità $k-2$ come somma di due addendi. In pratica stiamo calcolando la probabilità che il primo nodo infetti uno dei suoi vicini (quello a destra \textbf{oppure} quello a sinistra) \textbf{e} che i due rami indipendenti si propaghino fino a che la somma di nodi recovered sia \(k\).

Osserviamo che la \eqref{eq_1} vale per $k \in \lbrack 2,n-1 \rbrack $ ovvero fintantoché le dinamiche sulle due linee restano indipendenti. 

Procediamo ora al calcolo di $\mathbb{P}(K=1)$ e $\mathbb{P}(K=n)$.

La probabilità che un solo nodo sia colpito dalla malattia coincide con la probabilità che l'orologio di Poisson del primo nodo infetto si attivi prima degli orologi di Poisson dei due edge collegati.
Pertanto
  \[
		\mathbb{P}(K=1)=\mathbb{P}(X(\infty)= x_0 | X(0))=\frac{1}{1+2\beta}.
		      \]

       
  A questo punto ricordiamo che tutti gli stati assorbenti sono globalmente raggiungibili, di conseguenza
       \[
		\sum_{k=1}^{n} \mathbb{P}(K=k) =1
        \]
	è quindi immediato ricavare che
          \[
		\mathbb{P}(K=n)=1-\bigg(\mathbb{P}(K=1)+\sum_{k=2}^{n-1}\mathbb{P}(K=k) \bigg) =
        \]
         \[
		=1-\bigg(\frac{1}{1+2\beta}+\sum_{k=2}^{n-1}2(k-1)\frac{1}{1+2\beta}\frac{\beta^{k-1}}{(1+\beta)^k} \bigg).
        \]


	  









% Sometimes questions get separated from their bodies. Use a \newpage to force
% them to wrap to the next page.

  
% Use \renewcommand{\questiontype}{<text>} to change what word is displayed
% before numbered questions
%\renewcommand{\questiontype}{Task}
\end{document}
